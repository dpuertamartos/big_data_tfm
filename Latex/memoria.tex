\documentclass[a4paper,12pt,twoside]{memoir}

% Castellano
\usepackage[spanish,es-tabla]{babel}
\selectlanguage{spanish}
\usepackage[utf8]{inputenc}
\usepackage[T1]{fontenc}
\usepackage{lmodern} % Scalable font
\usepackage{microtype}
\usepackage{placeins}

\RequirePackage{booktabs}
\RequirePackage[table]{xcolor}
\RequirePackage{xtab}
\RequirePackage{multirow}

% Links
\usepackage[colorlinks]{hyperref}
\hypersetup{
	allcolors = {red}
}

% Ecuaciones
\usepackage{amsmath}

% Rutas de fichero / paquete
\newcommand{\ruta}[1]{{\sffamily #1}}

% Párrafos
\nonzeroparskip

% Huérfanas y viudas
\widowpenalty100000
\clubpenalty100000

% Evitar solapes en el header
\nouppercaseheads

% Imagenes
\usepackage{graphicx}
\newcommand{\imagen}[2]{
	\begin{figure}[!h]
		\centering
		\includegraphics[width=0.9\textwidth]{#1}
		\caption{#2}\label{fig:#1}
	\end{figure}
	\FloatBarrier
}

\newcommand{\imagenflotante}[2]{
	\begin{figure}%[!h]
		\centering
		\includegraphics[width=0.9\textwidth]{#1}
		\caption{#2}\label{fig:#1}
	\end{figure}
}



% El comando \figura nos permite insertar figuras comodamente, y utilizando
% siempre el mismo formato. Los parametros son:
% 1 -> Porcentaje del ancho de página que ocupará la figura (de 0 a 1)
% 2 --> Fichero de la imagen
% 3 --> Texto a pie de imagen
% 4 --> Etiqueta (label) para referencias
% 5 --> Opciones que queramos pasarle al \includegraphics
% 6 --> Opciones de posicionamiento a pasarle a \begin{figure}
\newcommand{\figuraConPosicion}[6]{%
  \setlength{\anchoFloat}{#1\textwidth}%
  \addtolength{\anchoFloat}{-4\fboxsep}%
  \setlength{\anchoFigura}{\anchoFloat}%
  \begin{figure}[#6]
    \begin{center}%
      \Ovalbox{%
        \begin{minipage}{\anchoFloat}%
          \begin{center}%
            \includegraphics[width=\anchoFigura,#5]{#2}%
            \caption{#3}%
            \label{#4}%
          \end{center}%
        \end{minipage}
      }%
    \end{center}%
  \end{figure}%
}

%
% Comando para incluir imágenes en formato apaisado (sin marco).
\newcommand{\figuraApaisadaSinMarco}[5]{%
  \begin{figure}%
    \begin{center}%
    \includegraphics[angle=90,height=#1\textheight,#5]{#2}%
    \caption{#3}%
    \label{#4}%
    \end{center}%
  \end{figure}%
}
% Para las tablas
\newcommand{\otoprule}{\midrule [\heavyrulewidth]}
%
% Nuevo comando para tablas pequeñas (menos de una página).
\newcommand{\tablaSmall}[5]{%
 \begin{table}
  \begin{center}
   \rowcolors {2}{gray!35}{}
   \begin{tabular}{#2}
    \toprule
    #4
    \otoprule
    #5
    \bottomrule
   \end{tabular}
   \caption{#1}
   \label{tabla:#3}
  \end{center}
 \end{table}
}

%
% Nuevo comando para tablas pequeñas (menos de una página).
\newcommand{\tablaSmallSinColores}[5]{%
 \begin{table}[H]
  \begin{center}
   \begin{tabular}{#2}
    \toprule
    #4
    \otoprule
    #5
    \bottomrule
   \end{tabular}
   \caption{#1}
   \label{tabla:#3}
  \end{center}
 \end{table}
}

\newcommand{\tablaApaisadaSmall}[5]{%
\begin{landscape}
  \begin{table}
   \begin{center}
    \rowcolors {2}{gray!35}{}
    \begin{tabular}{#2}
     \toprule
     #4
     \otoprule
     #5
     \bottomrule
    \end{tabular}
    \caption{#1}
    \label{tabla:#3}
   \end{center}
  \end{table}
\end{landscape}
}

%
% Nuevo comando para tablas grandes con cabecera y filas alternas coloreadas en gris.
\newcommand{\tabla}[6]{%
  \begin{center}
    \tablefirsthead{
      \toprule
      #5
      \otoprule
    }
    \tablehead{
      \multicolumn{#3}{l}{\small\sl continúa desde la página anterior}\\
      \toprule
      #5
      \otoprule
    }
    \tabletail{
      \hline
      \multicolumn{#3}{r}{\small\sl continúa en la página siguiente}\\
    }
    \tablelasttail{
      \hline
    }
    \bottomcaption{#1}
    \rowcolors {2}{gray!35}{}
    \begin{xtabular}{#2}
      #6
      \bottomrule
    \end{xtabular}
    \label{tabla:#4}
  \end{center}
}

%
% Nuevo comando para tablas grandes con cabecera.
\newcommand{\tablaSinColores}[6]{%
  \begin{center}
    \tablefirsthead{
      \toprule
      #5
      \otoprule
    }
    \tablehead{
      \multicolumn{#3}{l}{\small\sl continúa desde la página anterior}\\
      \toprule
      #5
      \otoprule
    }
    \tabletail{
      \hline
      \multicolumn{#3}{r}{\small\sl continúa en la página siguiente}\\
    }
    \tablelasttail{
      \hline
    }
    \bottomcaption{#1}
    \begin{xtabular}{#2}
      #6
      \bottomrule
    \end{xtabular}
    \label{tabla:#4}
  \end{center}
}

%
% Nuevo comando para tablas grandes sin cabecera.
\newcommand{\tablaSinCabecera}[5]{%
  \begin{center}
    \tablefirsthead{
      \toprule
    }
    \tablehead{
      \multicolumn{#3}{l}{\small\sl continúa desde la página anterior}\\
      \hline
    }
    \tabletail{
      \hline
      \multicolumn{#3}{r}{\small\sl continúa en la página siguiente}\\
    }
    \tablelasttail{
      \hline
    }
    \bottomcaption{#1}
  \begin{xtabular}{#2}
    #5
   \bottomrule
  \end{xtabular}
  \label{tabla:#4}
  \end{center}
}



\definecolor{cgoLight}{HTML}{EEEEEE}
\definecolor{cgoExtralight}{HTML}{FFFFFF}

%
% Nuevo comando para tablas grandes sin cabecera.
\newcommand{\tablaSinCabeceraConBandas}[5]{%
  \begin{center}
    \tablefirsthead{
      \toprule
    }
    \tablehead{
      \multicolumn{#3}{l}{\small\sl continúa desde la página anterior}\\
      \hline
    }
    \tabletail{
      \hline
      \multicolumn{#3}{r}{\small\sl continúa en la página siguiente}\\
    }
    \tablelasttail{
      \hline
    }
    \bottomcaption{#1}
    \rowcolors[]{1}{cgoExtralight}{cgoLight}

  \begin{xtabular}{#2}
    #5
   \bottomrule
  \end{xtabular}
  \label{tabla:#4}
  \end{center}
}


\graphicspath{ {./img/} }

% Capítulos
\chapterstyle{bianchi}
\newcommand{\capitulo}[2]{
	\setcounter{chapter}{#1}
	\setcounter{section}{0}
	\chapter*{#2}
	\addcontentsline{toc}{chapter}{#1. #2}
	\markboth{#2}{#2}
}

% Apéndices
\renewcommand{\appendixname}{Apéndice}
\renewcommand*\cftappendixname{\appendixname}

\newcommand{\apendice}[1]{
	%\renewcommand{\thechapter}{A}
	\chapter{#1}
}

\renewcommand*\cftappendixname{\appendixname\ }

% Formato de portada
\makeatletter
\usepackage{xcolor}
\newcommand{\tutor}[1]{\def\@tutor{#1}}
\newcommand{\course}[1]{\def\@course{#1}}
\definecolor{cpardoBox}{HTML}{E6E6FF}
\def\maketitle{
  \null
  \thispagestyle{empty}
  % Cabecera ----------------
\begin{center}%
	{\noindent\Huge Universidades de Burgos, León y Valladolid}\vspace{.5cm}%
	
	{\noindent\Large Máster universitario}\vspace{.5cm}%
	
	{\noindent\Huge \textbf{Inteligencia de Negocio y Big~Data en Entornos Seguros}}\vspace{.5cm}%
\end{center}%

\begin{center}%
	\includegraphics[height=3cm]{img/escudoUBU} \hspace{1cm}
	\includegraphics[height=3cm]{img/escudoUVA} \hspace{1cm}
	\includegraphics[height=3cm]{img/escudoULE} \vspace{1cm}%
\end{center}%

  \vfill
  % Título proyecto y escudo informática ----------------
  \colorbox{cpardoBox}{%
    \begin{minipage}{.9\textwidth}
      \vspace{.5cm}\Large
      \begin{center}
      \textbf{TFM del Máster Inteligencia de Negocio y Big Data en Entornos Seguros}\vspace{.6cm}\\
      \textbf{\LARGE\@title{}}
      \end{center}
      \vspace{.2cm}
    \end{minipage}

  }%
  \hfill
  \vfill
  % Datos de alumno, curso y tutores ------------------
  \begin{center}%
  {%
    \noindent\LARGE
    Presentado por \@author{}\\ 
    en Universidad de Burgos --- \@date{}\\
    Tutor: \@tutor{}\\
  }%
  \end{center}%
  \null
  \cleardoublepage
  }
\makeatother

\newcommand{\nombre}{Nombre del alumno} %%% cambio de comando

% Datos de portada
\title{título del TFM}
\author{\nombre}
\tutor{nombre tutor}
\date{\today}

\begin{document}

\maketitle


\newpage\null\thispagestyle{empty}\newpage


%%%%%%%%%%%%%%%%%%%%%%%%%%%%%%%%%%%%%%%%%%%%%%%%%%%%%%%%%%%%%%%%%%%%%%%%%%%%%%%%%%%%%%%%
\thispagestyle{empty}


\noindent
\begin{center}%
	{\noindent\Huge Universidades de Burgos, León y Valladolid}\vspace{.5cm}%
	
\begin{center}%
	\includegraphics[height=3cm]{img/escudoUBU} \hspace{1cm}
	\includegraphics[height=3cm]{img/escudoUVA} \hspace{1cm}
	\includegraphics[height=3cm]{img/escudoULE} \vspace{1cm}%
\end{center}%

	{\noindent\Large \textbf{Máster universitario en Inteligencia de Negocio y Big~Data en Entornos Seguros}}\vspace{.5cm}%
\end{center}%



\noindent D. nombre tutor, profesor del departamento de nombre departamento, área de nombre área.

\noindent Expone:

\noindent Que el alumno D. \nombre, con DNI dni, ha realizado el Trabajo final de Máster en Inteligencia de Negocio y Big Data en Entornos Seguros 
          titulado título de TFM. 

\noindent Y que dicho trabajo ha sido realizado por el alumno bajo la dirección del que suscribe, en virtud de lo cual se autoriza su presentación y defensa.

\begin{center} %\large
En Burgos, {\large \today}
\end{center}

\vfill\vfill\vfill

% Author and supervisor
\begin{minipage}{0.45\textwidth}
\begin{flushleft} %\large
Vº. Bº. del Tutor:\\[2cm]
D. nombre tutor
\end{flushleft}
\end{minipage}
\hfill
\begin{minipage}{0.45\textwidth}
\begin{flushleft} %\large
Vº. Bº. del co-tutor:\\[2cm]
D. nombre co-tutor
\end{flushleft}
\end{minipage}
\hfill

\vfill

% para casos con solo un tutor comentar lo anterior
% y descomentar lo siguiente
%Vº. Bº. del Tutor:\\[2cm]
%D. nombre tutor


\newpage\null\thispagestyle{empty}\newpage




\frontmatter

% Abstract en castellano
\renewcommand*\abstractname{Resumen}
\begin{abstract}
En este primer apartado se hace una \textbf{breve} presentación del tema que se aborda en el proyecto.
\end{abstract}

\renewcommand*\abstractname{Descriptores}
\begin{abstract}
Palabras separadas por comas que identifiquen el contenido del proyecto Ej: servidor web, buscador de vuelos, android \ldots
\end{abstract}

\clearpage

% Abstract en inglés
\renewcommand*\abstractname{Abstract}
\begin{abstract}
A \textbf{brief} presentation of the topic addressed in the project.
\end{abstract}

\renewcommand*\abstractname{Keywords}
\begin{abstract}
keywords separated by commas.
\end{abstract}

\clearpage

% Indices
\tableofcontents

\clearpage

\listoffigures

\clearpage

\listoftables
\clearpage

\mainmatter

\part*{Memoria}
\addcontentsline{toc}{part}{Memoria}


\capitulo{1}{Introducción}

En la era digital actual, el mercado de inversiones inmobiliarias en España está experimentando una transformación sin precedentes, principalmente por una combinación de factores económicos, sociales y tecnológicos. La búsqueda de inmuebles para comprar se ha vuelto cada vez más desafiante en un contexto donde los precios han visto un encarecimiento notable en los últimos años, lo que ha generado preocupaciones tanto para inversores como para aquellos que simplemente buscan un lugar para vivir. 

Según Idealista, el portal inmobiliario líder en España, el año 2023 terminará con un balance aparentemente contradictorio: mientras que las hipotecas disminuyen y el volumen de operaciones se resiente tras un año relevante como fue 2022, los precios de las viviendas continúan al alza, impulsados por una demanda que aún supera a una oferta cada vez más limitada~\cite{idealista_study}.

Este proyecto busca abordar precisamente estos desafíos, aplicando procesos de Ingeniería de Datos y técnicas de Aprendizaje Automático para optimizar la toma de decisiones en el ámbito de las inversiones inmobiliarias. Se pretende ofrecer una herramienta que no solo sea capaz de identificar oportunidades de inversión o búsqueda de nuevo hogar, sino también estudiar las fluctuaciones del mercado.

El proyecto se estructura en varias fases críticas, comenzando con la adquisición y preparación de datos a través de técnicas de \textit{scraping} y procesos ETL, seguido de la implementación de modelos de Aprendizaje Automático para evaluar el valor de las propiedades inmobiliarias en venta. La integración de estos datos enriquecidos en una aplicación web interactiva permite a los usuarios identificar inversiones inmobiliarias atractivas, además de estudiar la evolución del mercado inmobiliario.

Esta memoria también aborda los desafíos técnicos y metodológicos encontrados durante el desarrollo del proyecto, desde la selección de la fuente de datos hasta la optimización de los modelos de Aprendizaje Automático y la implementación de la solución en un entorno de producción. Además, se explora la relevancia de las tecnologías empleadas, remarcando por qué se han considerado para el proyecto.

Al finalizar, se presentan las conclusiones derivadas del trabajo, así como posibles líneas de investigación y desarrollo futuro para expandir y enriquecer aún más la solución propuesta. Por último, se incluyen Apéndices como el plan de proyecto, la guía de diseño de datos y arquitectura, el manual de programador y manual de usuario del sitio web resultante, disponible en \href{https://www.buscahogar.es}{buscahogar.es}.
\capitulo{2}{Objetivos del proyecto}
Este proyecto tiene como objetivo principal desarrollar una solución de ingeniería de datos e inteligencia de negocio para descubrimiento de oportunidades inmobiliarias, para ello se establecen las siguientes metas:

\begin{enumerate}

\item Desarrollo de un \textit{\textit{scraper}} para la sección de compra de inmuebles de \url{www.pisos.com} :

Este objetivo implica el desarrollo de un software que acceda periódicamente a la sección de compra-venta de inmuebles del portal inmobiliario y recopile información actualizada de dichos inmuebles a la venta. Se pretende acceder a datos de todas las provincias de España en orden de publicación (más reciente a menos).

\item Almacenamiento de los datos obtenidos por el \textit{scraper} en una base de datos no relacional:

El software de \textit{scraping}, tras la recopilación de datos de cada inmueble los cargará en diversas colecciones (una por provincia) de una base de datos no relacional. Se guardarán para cada inmueble la mayor cantidad posible de atributos.

\item Transformación y agregación de los datos para su posterior uso:

Se pretende desarrollar un flujo de datos que extraiga los datos de la base de datos no relacional, los transforme, limpie y además los agregue según distintas dimensiones: territoriales, temporales y según si el anuncio sigue activo.

\clearpage
\item Carga de los datos transformados en una base de datos relacional:

Los datos transformados y agregados se cargarán en formato tabular en una base de datos relacional que admita consultas SQL.

\item Entrenamiento de modelos de Aprendizaje Automático utilizando los datos almacenados. Enriquecimiento de los datos de la base de datos relacional mediante el uso de dichos modelos:

Un objetivo principal es utilizar los datos transformados y limpiados para entrenar modelos de Aprendizaje Automático que permitan asignar una puntuación a cada inmueble que permita aproximar cómo de interesante resulta ese inmueble.

\item Desarrollo de una web donde se puedan acceder y visualizar los datos transformados, agregados y enriquecidos:

Se pretende mostrar los datos recopilados por el flujo en una interfaz web, en esta web también se mostraran las puntuaciones aportadas por los modelos de Aprendizaje Automático. Además permitirá la visualización de los datos agregados.

\item Orquestar el proceso de manera que los datos se extraigan, carguen, transformen y analicen de manera prolongada en el tiempo:

Se pretende que el flujo de datos se orqueste de manera que funcione de forma automatizada y continúa durante todo el proyecto, ofreciendo datos actualizados a diario.

\end{enumerate}
\capitulo{3}{Conceptos teóricos}

En aquellos proyectos que necesiten para su comprensión y desarrollo de unos conceptos teóricos de una determinada materia o de un determinado dominio de conocimiento, debe existir un apartado que sintetice dichos conceptos.

Algunos conceptos teóricos de \LaTeX \footnote{Créditos a los proyectos de Álvaro López Cantero: Configurador de Presupuestos y Roberto Izquierdo Amo: PLQuiz}.

\section{Scrapping}

Las secciones se incluyen con el comando section.

\section{ETL}

Las secciones se incluyen con el comando section.

\section{Aprendizaje automático de datos inmobiliarios}

Las secciones se incluyen con el comando section.

\subsection{Subsecciones}

Además de secciones tenemos subsecciones.

\subsubsection{Subsubsecciones}

Y subsecciones. 


\section{Referencias}

Las referencias se incluyen en el texto usando cite \cite{wiki:latex}. Para citar webs, artículos o libros \cite{koza92}.


\section{Imágenes}

Se pueden incluir imágenes con los comandos standard de \LaTeX, pero esta plantilla dispone de comandos propios como por ejemplo el siguiente:

\imagen{escudoInfor}{Autómata para una expresión vacía}



\section{Listas de items}

Existen tres posibilidades:

\begin{itemize}
	\item primer item.
	\item segundo item.
\end{itemize}

\begin{enumerate}
	\item primer item.
	\item segundo item.
\end{enumerate}

\begin{description}
	\item[Primer item] más información sobre el primer item.
	\item[Segundo item] más información sobre el segundo item.
\end{description}
	
\begin{itemize}
\item 
\end{itemize}

\section{Tablas}

Igualmente se pueden usar los comandos específicos de \LaTeX o bien usar alguno de los comandos de la plantilla.

\tablaSmall{Herramientas y tecnologías utilizadas en cada parte del proyecto}{l c c c c}{herramientasportipodeuso}
{ \multicolumn{1}{l}{Herramientas} & App AngularJS & API REST & BD & Memoria \\}{ 
HTML5 & X & & &\\
CSS3 & X & & &\\
BOOTSTRAP & X & & &\\
JavaScript & X & & &\\
AngularJS & X & & &\\
Bower & X & & &\\
PHP & & X & &\\
Karma + Jasmine & X & & &\\
Slim framework & & X & &\\
Idiorm & & X & &\\
Composer & & X & &\\
JSON & X & X & &\\
PhpStorm & X & X & &\\
MySQL & & & X &\\
PhpMyAdmin & & & X &\\
Git + BitBucket & X & X & X & X\\
Mik\TeX{} & & & & X\\
\TeX{}Maker & & & & X\\
Astah & & & & X\\
Balsamiq Mockups & X & & &\\
VersionOne & X & X & X & X\\
} 

\capitulo{4}{Técnicas y herramientas}

Esta parte de la memoria tiene como objetivo presentar las técnicas metodológicas y las herramientas de desarrollo que se han utilizado para llevar a cabo el proyecto. Si se han estudiado diferentes alternativas de metodologías, herramientas, bibliotecas se puede hacer un resumen de los aspectos más destacados de cada alternativa, incluyendo comparativas entre las distintas opciones y una justificación de las elecciones realizadas. 
No se pretende que este apartado se convierta en un capítulo de un libro dedicado a cada una de las alternativas, sino comentar los aspectos más destacados de cada opción, con un repaso somero a los fundamentos esenciales y referencias bibliográficas para que el lector pueda ampliar su conocimiento sobre el tema.

\section{Tecnología de scrapping: Scrappy}


En el ámbito del web scraping existen múltiples herramientas y bibliotecas que facilitan la extracción de datos de sitios web para su posterior análisis o almacenamiento. La Tabla \ref{tabla:tecnologiasComparadas} compara algunas de las tecnologías de web scraping más conocidas, evaluándolas en base a varios criterios como el rendimiento, facilidad de uso, y flexibilidad.

Para el proyecto actual, que tiene como objetivo extraer información relevante sobre la compraventa de inmuebles en pisos.com, se ha seleccionado Scrapy como la tecnología de scraping a utilizar. A continuación, se describen las razones de esta elección:

\begin{description}
	\item[Rendimiento] Scrapy es conocido por su alta velocidad y eficiencia, permitiendo la extracción de grandes cantidades de datos en un tiempo reducido. Esto es especialmente útil para nuestro proyecto, donde la actualización frecuente de los anuncios y las ofertas es una constante.
	\item[Facilidad de Uso] Aunque Scrapy tiene una curva de aprendizaje inicial más pronunciada en comparación con, por ejemplo, Beautiful Soup, ofrece una gran cantidad de funcionalidades "out-of-the-box" que aceleran el proceso de desarrollo una vez se comprende su funcionamiento básico.
    \item[Flexibilidad:] Scrapy es altamente configurable y extensible, lo cual permite adaptar el scraper a necesidades específicas. Esto resulta particularmente útil cuando se trata de sitios web con estructuras más complejas o cuando se requieren funcionalidades avanzadas como el manejo de sesiones, cookies o cabeceras HTTP.
    \item[Madurez y Comunidad:] Scrapy es una tecnología madura con una comunidad activa, lo que asegura un buen soporte y una amplia disponibilidad de documentación y recursos de aprendizaje.
    \item[Preferencia del autor para aprender la herramienta:] Algunas de las herramientas como Beautiful Soup y Selenium ya habían sido usadas por el autor, por lo tanto uno de los motivos para decidir por Scrapy fue el hecho de aprender una nueva herramienta ampliamente usada en la comunidad.
\end{description}

Por todas estas razones, Scrapy se presenta como la opción más robusta y versátil para el web scraping de \url{www.pisos.com}, proporcionando las herramientas necesarias para llevar a cabo un proyecto exitoso.



\tablaSmall{Comparación de Tecnologías de Web Scraping}{l c c c c}{tecnologiasComparadas}
{ \multicolumn{1}{l}{Tecnologías} & Rendimiento & Facilidad de Uso & Flexibilidad & Proyecto \\}{
Scrapy & \textbf{XX} & X & \textbf{XX} & \textbf{XX}\\
Beautiful Soup & X & \textbf{XX} & X & \\
Selenium & X & X & X & \\
Requests & X & \textbf{XX} & X & \\
Mechanize & X & X & X & \\
}

\section{Base de datos primaria: MongoDB}

Los datos recopilados a través de scrapping deben almacenarse de forma permanente para su posterior uso. Esto requiere inevitablemente usar una base de datos.

Una de las primeras decisiones fue usar una base de datos no relacional, por la flexibilidad que podía aportar al tratarse de datos de scrapping. Es frecuente que estos datos no tengan una estructura muy definida o que varíen considerablemente en formato, campos disponibles o tipos de datos. En tales casos, las bases de datos relacionales pueden resultar restrictivas, ya que exigen un esquema fijo y predefinido que todos los registros deben seguir. Por el contrario, MongoDB permite una estructura más flexible, lo que hace que sea más adecuada para manejar datos heterogéneos.

Dentro de las bases de datos NoSQL, otra decisión rápida fue la de elegir una que fuera Open Source y con licencia gratuita. Esto no solo ayuda a mantener bajos los costos del proyecto, sino que también abre la puerta a una amplia comunidad de desarrolladores y una gran cantidad de recursos y documentación en línea. De las bases de datos que cumplían con estos requisitos, MongoDB destacó por varias razones:

\begin{itemize}
\item \textbf{Prevalencia en la Industria}: MongoDB es una de las bases de datos NoSQL más populares y ampliamente utilizadas. Esto no solo la hace una tecnología atractiva para aprender desde una perspectiva de desarrollo profesional, sino que también asegura una amplia gama de soporte comunitario y empresarial.

\item \textbf{Experiencia Previa}: Contar con experiencia previa en MongoDB reduce significativamente la curva de aprendizaje, permitiendo un desarrollo más rápido y eficiente del proyecto.

\item \textbf{Escalabilidad}: MongoDB ofrece una excelente escalabilidad horizontal, lo que permite manejar grandes volúmenes de datos y tráfico sin degradar el rendimiento. Esto es especialmente útil para proyectos de scrapping que pueden empezar pequeños pero crecer rápidamente.

\item \textbf{Consultas Flexibles}: MongoDB ofrece un sistema de consultas flexible y potente que permite realizar búsquedas complejas, algo que es especialmente útil cuando se trata de analizar y utilizar los datos recopilados.

\item \textbf{Integración con otras Tecnologías}: MongoDB se integra fácilmente con numerosas plataformas y lenguajes de programación, lo que la convierte en una opción versátil para cualquier stack tecnológico. La integración son scrapy, la libreria de python para web scrapping utilizada fue trivial gracias a la libreria pymongo, la libreria de mongodb para python.
\end{itemize}

Por todas estas razones, MongoDB se convirtió en la opción más atractiva para este proyecto, ofreciendo la combinación ideal de flexibilidad, escalabilidad y soporte comunitario.

\section{Base de datos secundaria: SQLite}

Los datos originados de la base de datos primaria y transformados mediante el proceso ETL (Extracción, Transformación y Carga) son nuevamente almacenados en una base de datos. Para este paso del flujo de datos, se optó por una base de datos relacional como SQLite por diversas razones:

\begin{itemize}
\item \textbf{Búsquedas y Ordenaciones Rápidas}: Las bases de datos relacionales son particularmente eficientes en la realización de consultas que involucran ordenamientos y joins, lo que resulta útil para las interfaces web donde se necesita acceder a los datos de forma rápida y eficiente.

\item \textbf{Estructura Clara}: Este tipo de base de datos proporciona un esquema bien definido, lo cual facilita el análisis de datos y las operaciones de machine learning, al no requerir transformaciones adicionales para su uso.

\item \textbf{Economía de Recursos}: Dado que el volumen de datos a manejar es relativamente pequeño (inferior a 500 MB cada 2-3 meses de recolección), una base de datos ligera y eficiente como SQLite es más que suficiente para satisfacer las necesidades del proyecto.

\item \textbf{Integración con Python}: SQLite permite una integración directa con Python, lo que simplifica significativamente el flujo de trabajo y evita la necesidad de implementar y mantener un servidor de base de datos separado.

\item \textbf{Facilidad de Migración}: En caso de que el proyecto escale y requiera una solución más robusta, la migración a una base de datos relacional más potente, como PostgreSQL o MySQL, sería un proceso relativamente sencillo. Esto es debido a que el esquema y las consultas SQL podrían reutilizarse con pocos ajustes.
\end{itemize}

Por lo tanto, SQLite se convierte en una excelente opción para este escenario específico. Ofrece la ventaja de ser ligero y eficiente en el uso de recursos, mientras proporciona todas las funcionalidades necesarias para realizar análisis de datos y machine learning de forma eficaz. Además, su fácil integración con Python y la flexibilidad para una futura migración hacen de SQLite una elección pragmática y efectiva para este proyecto.

\section{Herramientas de ETL: Pandas}

Para el proceso de Extracción, Transformación y Carga (ETL) de los datos, se decidió utilizar Python con la ayuda de la librería Pandas. A continuación, se describen las razones que llevaron a esta elección:

\begin{itemize}
\item \textbf{Volumen de Datos}: Uno de los principales factores fue el tamaño moderado del conjunto de datos con el que se está trabajando. Con aproximadamente 500 MB de datos recopilados cada 2-3 meses, el volumen no justifica el uso de una herramienta diseñada para el procesamiento de datos masivos, como Spark.

\item \textbf{Eficiencia y Velocidad}: Con Pandas y Python, la transformación de toda la base de datos se puede realizar en cuestión de segundos. Esto significa que no hay una necesidad inmediata de una infraestructura más compleja y potente. Usar Spark en este contexto sería una forma de sobreingeniería que añadiría complejidad innecesaria al proyecto.

\item \textbf{Recursos de Máquina}: Spark exige una asignación de recursos considerablemente mayor que Pandas, especialmente si se configura en un modo distribuido. Estos recursos podrían ser más eficientemente dedicados a otras tecnologías o aspectos del proyecto.

\item \textbf{Facilidad y flexibilidad}: Pandas ofrece una API intuitiva y fácil de usar, lo que acelera el desarrollo y facilita el mantenimiento. Además, la comunidad de Pandas es amplia, con una gran cantidad de documentación y tutoriales disponibles. Pandas es extremadamente flexible y permite una amplia gama de operaciones de manipulación de datos, desde simples filtrados y ordenaciones hasta operaciones de agrupación y pivoteo más complejas.

\item \textbf{Costo y Licencia}: Al ser una biblioteca de código abierto, Pandas no incurre en costos adicionales, lo cual es beneficioso desde el punto de vista económico del proyecto.
\end{itemize}

Por todas estas razones, se optó por utilizar Python con Pandas para las operaciones de ETL en este proyecto. Esta elección se alinea con los requisitos y limitaciones del proyecto, ofreciendo una solución que es tanto eficiente como efectiva, sin incurrir en la complejidad y los costos adicionales que implicaría la implementación de una herramienta como Spark.

\section{Orquestador: Apache Airflow}

Originalmente, la ingestión y transformación de datos se manejaban con una simple programación cron. Sin embargo, con la evolución del proyecto y la incorporación de nuevos flujos de trabajo, como el entrenamiento periódico de modelos de machine learning y la aplicación de dichos modelos a nuevos datos, se hizo evidente la necesidad de un sistema de orquestación más robusto y flexible. Por ello, se optó por utilizar Apache Airflow por las siguientes razones:

\begin{itemize}
\item \textbf{Gestión de Dependencias}: Apache Airflow permite definir de manera clara y estructurada las dependencias entre las diferentes tareas del flujo de trabajo. Esto resulta especialmente útil cuando los flujos de trabajo se vuelven más complejos y dependen de múltiples etapas y condiciones para su ejecución exitosa.

\item \textbf{Interfaz de Usuario}: Apache Airflow viene con una interfaz de usuario intuitiva que facilita el monitoreo del estado de los flujos de trabajo, la revisión de logs y la identificación de cuellos de botella o fallos en el sistema.

\item \textbf{Programación Flexible}: A diferencia de cron, que tiene limitaciones en cuanto a la programación de tareas, Airflow permite una gran flexibilidad en la definición de horarios y desencadenantes para la ejecución de tareas.

\item \textbf{Integración con Otras Herramientas}: Airflow se integra fácilmente con una amplia variedad de tecnologías y plataformas, desde bases de datos hasta servicios de almacenamiento en la nube, lo que facilita la implementación de flujos de trabajo complejos.

\item \textbf{Gestión de Errores y Reintentos}: Con Airflow, es posible configurar políticas de reintentos y alertas, lo que mejora la robustez del sistema al manejar fallos y excepciones de manera más efectiva.

\item \textbf{Código Como Configuración}: Airflow permite definir flujos de trabajo como código, lo que facilita la versión, el mantenimiento y la colaboración en el desarrollo de flujos de trabajo.

\item \textbf{Comunidad y Soporte}: Apache Airflow cuenta con una comunidad de desarrolladores activa y una gran cantidad de documentación en línea, lo que facilita el proceso de adaptación y resolución de problemas.
\end{itemize}

Por lo aquí expuesto, Apache Airflow fue elegido como la solución de orquestación más adecuada para este proyecto. Su flexibilidad, escalabilidad y robustez hacen que sea una herramienta altamente eficaz para coordinar múltiples tareas, algo que cron simplemente no podría manejar de manera tan efectiva.


\section{Machine learning: scikit learn}

Las secciones se incluyen con el comando section.

\section{Aplicación web: Frontend - React}

Las secciones se incluyen con el comando section.

\section{Aplicación web: Backend - Node.js, express}

Las secciones se incluyen con el comando section.

\section{Despliegue - Oracle cloud}

Las secciones se incluyen con el comando section.

\capitulo{5}{Aspectos relevantes del desarrollo del proyecto}

Este apartado pretende recoger los aspectos más interesantes del desarrollo del proyecto, comentados por los autores del mismo.
Debe incluir desde la exposición del ciclo de vida utilizado, hasta los detalles de mayor relevancia de las fases de análisis, diseño e implementación.
Se busca que no sea una mera operación de copiar y pegar diagramas y extractos del código fuente, sino que realmente se justifiquen los caminos de solución que se han tomado, especialmente aquellos que no sean triviales.
Puede ser el lugar más adecuado para documentar los aspectos más interesantes del diseño y de la implementación, con un mayor hincapié en aspectos tales como el tipo de arquitectura elegido, los índices de las tablas de la base de datos, normalización y desnormalización, distribución en ficheros3, reglas de negocio dentro de las bases de datos (EDVHV GH GDWRV DFWLYDV), aspectos de desarrollo relacionados con el WWW...
Este apartado, debe convertirse en el resumen de la experiencia práctica del proyecto, y por sí mismo justifica que la memoria se convierta en un documento útil, fuente de referencia para los autores, los tutores y futuros alumnos.

\section{Diseño inicial}

\section{Elección de la fuente de datos}

\section{Elección de tecnología ETL}

\section{Mejora del sistema de orquestación}
\capitulo{6}{Trabajos relacionados}

Este apartado sería parecido a un estado del arte de una tesis o tesina. En un trabajo final de máster no parece tan obligada su presencia, aunque se puede dejar a juicio del tutor el incluir un pequeño resumen comentado de los trabajos y proyectos ya realizados en el campo del proyecto en curso. 

https://github.com/AdrianRiesco/Masters-Thesis-on-Big-Data/blob/main/doc/memoria.pdf

https://github.com/jlgarridol/TFM-FIS-IF/blob/master/doc/memoria.pdf

https://github.com/Josemi/TFM-FIS-IA/blob/master/doc/Latex/memoria.pdf

https://github.com/IvanBeke/TFM/blob/main/memoria/memoria.pdf
\capitulo{7}{Conclusiones y Líneas de trabajo futuras}


Este proyecto ha demostrado la viabilidad y el potencial de aplicar procesos ETL y técnicas de Aprendizaje Automático para guiar la toma de decisiones en inversiones inmobiliarias. A través de un enfoque sistemático y la utilización de tecnologías de código abierto, se ha logrado desarrollar una solución que facilita el análisis de las tendencias del mercado inmobiliario y además ofrece una guía de predicción del valor de las propiedades inmobiliarias en forma de una web interactiva y accesible para cualquier persona.

\subsection*{Hallazgos Clave:}

Los principales hallazgos del proyecto se pueden resumir en los siguientes puntos:

\begin{itemize}
  \item La simple extracción, transformación y agregación de los datos, permite crear visualizaciones que aportan conclusiones interesantes sobre el mercado inmobiliario.
  \item La implementación de modelos de Aprendizaje Automático específicos para diferentes rangos de precios y provincias ha permitido obtener predicciones más precisas y adaptadas a las particularidades del mercado inmobiliario español.
  \item La utilización de Docker y Oracle Cloud ha proporcionado una infraestructura robusta, escalable y eficiente, esencial para el manejo de los numerosos servicios que intervienen en todo el flujo de datos.
  \item Aunque los modelos presentan errores significativos en ciertos casos, es posible aplicar un método de filtrado efectivo mediante el uso de un criterio simple: descartar aquellos inmuebles a los cuales el algoritmo asigna un valor aproximadamente un 33-40\% superior al precio de venta. Esta sobrevaloración generalmente se debe a la falta de información en el anuncio, errores en los datos proporcionados, o a características del inmueble (como aspectos negativos visibles en fotos o desventajas de la ubicación) que el modelo no logra interpretar adecuadamente.

\end{itemize}

\subsection*{Limitaciones y Desafíos:}

A continuación se muestran las principales limitaciones encontradas hasta la fecha:

\begin{itemize}
  \item La recopilación de datos encontró obstáculos significativos debido a las limitaciones de acceso y las medidas anti-\textit{scraping} implementadas por algunos portales inmobiliarios. 
  \item El constante cambio del portal inmobiliario de origen obliga a continuamente adaptar la solución software de \textit{scraping} a dichos cambios. 
  \item La escalabilidad del proyecto se ve limitada por la capacidad de la infraestructura actual: Una sola máquina virtual.
  \item Los modelos utilizados son totalmente incapaces de interaccionar elementos muy importantes para categorizar el valor: Las fotografías y la descripción. En ellas se pueden ver muchas sutilezas y datos que los seres humanos utilizamos para dar valor a un inmueble.
\end{itemize}

\clearpage
\subsection*{Líneas de Trabajo Futuras:}

Las principales líneas de trabajo futuras que se consideran interesantes son las siguientes:

\begin{enumerate}
  \item \textbf{Expansión de la Fuente de datos:} Explorar nuevos portales inmobiliarios y negociar accesos a APIs para enriquecer el conjunto de datos, mejorar la precisión de los modelos predictivos y eliminar las limitaciones provenientes del uso de \textit{scraping}.
  \item \textbf{Optimización de Modelos:} Actualmente no se ajustan los hiperparámetros de los modelos, simplemente se prueban distintos algoritmos con los parámetros base y se escoge el que menor error relativo ofrece. Un enfoque inmediato es dedicar una máquina exclusivamente al entrenamiento de los modelos, lo que permitirá dedicar muchas más horas de computación a ajustar los parámetros, sin afectar al resto de servicios. 
  \item \textbf{Mejora de la Infraestructura:} Esta mejora abriría un gran abanico de posibilidades:

   Permitiría un \textbf{procesamiento de datos más frecuente} y ``casi en tiempo real''. Se podría hacer una actualización del flujo de datos cada hora, dando una sensación casi en tiempo real para el usuario, para ello sería esencial tener un servidor dedicado exclusivamente a ETL.

   Se podría expandir notoriamente la \textbf{capacidad de usuarios concurrentes} usando la web: Se dedicaría un servidor exclusivamente a la web y otro exclusivamente a la base de datos relacional, que debería migrarse a un sistema más apto para producción que SQLite como por ejemplo Postgres o MySQL.
  
  \item \textbf{Aumentar la interacción del usuario con los modelos:} Una expansión interesante de las capacidades del producto final, es permitir a los usuarios interactuar directamente con los modelos. Los usuarios podrían introducir las características que desean para su ``inmueble'' y el modelo les devolvería el valor medio de mercado esperado para un inmueble de esas características, esto sería útil tanto para saber por cuanto vender un inmueble de forma ajustada con el mercado, como para saber por cuanto podrían esperar comprar en el mercado un inmueble de las características señaladas.

  \item \textbf{Expandir el uso de Inteligencia Artificial:} Integrar tecnologías avanzadas de Inteligencia Artificial a los modelos de regresión actuales podría enriquecer significativamente el análisis. La implementación de algoritmos capaces de analizar fotografías para detectar características relevantes, identificar posibles errores en los anuncios, y mejorar la clasificación de inmuebles mediante el procesamiento del lenguaje natural (por ejemplo, analizando las descripciones de los anuncios) podría incrementar notablemente la precisión y profundidad del análisis de propiedades. Por ejemplo, en el artículo de Niu et al.~\cite{niu2019}, se utiliza un enfoque combinado de identificación de inmuebles repetidos (provenientes de distintos portales), selección de características a usar para entrenar en los modelos y evaluación simultánea con distintos modelos. Esta aproximación supone un interesante punto de partida para la mejora del algoritmo de evaluación de inmuebles del actual proyecto.


\end{enumerate}





%\renewcommand\chaptername{Anexo}
%\renewcommand\thechapter{\Roman{chapter}}
%\setcounter{chapter}{0}

% Añadir entrada en el índice: Anexos
\appendix
\addcontentsline{toc}{part}{Apéndices}
\part*{Apéndices}

\apendice{Plan de Proyecto Software}

\section{Introducción}

En este apartado se pretende exponer la planificación del proyecto. Se mostrará el avance temporal del proyecto, y los avances que tuvieron lugar entre cada reunión.

\section{Planificación temporal}

Se ha intentado seguir la metodología Agile, utilizando sprints o bloques de trabajo de unas dos semanas. En ocasiones por circunstancias de trabajo, vacaciones o imposibilidad los sprints se han prolongado durante 1 mes.

Tras cada sprint se mantenía una reunión entre los tutores y alumno, en la cual se debatian los avances y se planificaba en qué se iba a trabajar en el siguiente sprint.

\subsection{Sprint 1 - 27/06/23 a 5/07/23}

En este sprint se hizo un primer estudio de la viabilidad del proyecto y se decidieron las tecnologías a utilizar según la idea propuesta originalmente. Además se hizo un prototipo del scrapper.

\begin{itemize}
    \item Se comprobó que la página web \url{www.idealista.com} estaba totalmente protegida contra el scrapping. Tras pedir una clave para API que permitiera el desarrollo del proyecto no se obtuvo respuesta por parte de idealista.

    Por tanto, se decidió que la fuente de los datos iba a ser \url{www.pisos.com}, otro portal inmobiliario de los más grandes de españa y que no bloquea el uso de scrapping.
    \item Se hizo un diagrama de las tecnológías que se pretendian usar para la fase de extracción, transformación, carga de los datos y su análisis mediante machine learning, además de la orquestación de los distintos procesos.
    \item Se probaron diversas tecnologías de scrapping, para finalmente decidir usar scrappy
    \item Se desarrolló un prototipo de scrapper para inmuebles en venta en \url{www.pisos.com}
\end{itemize}

\subsection{Sprint 2 - 05/07/23 a 29/08/23}

En este sprint, algo más largo por las fechas vacacionales, se logró un gran avance en la parte de desarrollo.

\begin{itemize}
    \item Se amplió el software de scrapping, pasando a recopilar más de 50 datos por inmueble.
    \item Se diseñó y desplegó la base de datos primaria que almacenaria la ingestión de datos proveniente de scrapping: mongodb.
    \item Se desplegó una máquina virtual ubuntu alojada en Oracle cloud, en la cual se empezaron a ingestar datos diariamente.
    \item Se estableció un mecanismo de backup de la base de datos primaria en caso de catástrofe en la máquina virtual del proyecto.
    \item Se comenzó a esbozar el procedimiento de ETL de datos raw almacenados en la mongodb.
\end{itemize}

\subsection{Sprint 3 - 29/08/23 - 13/09/23}

En este sprint, se desplegó el proceso de ETL.

\begin{itemize}
    \item Finalización del diseño y decisión de las tecnologías aplicadas en el proceso de ETL.
    \item Se desplegó el proceso de ETL, en python, que recoge los datos raw de la mongodb, los transforma y los carga, en una base de datos relacional: SQLlite.
    \item Se hicieron algunas exploraciones previas de los datos limpios.
\end{itemize}

\subsection{Sprint 4 -  13/09/23 - 26/09/23}

Este sprint se dedicó a la puesta al día de la documentación del proyecto. Además se desplegó apache airflow como orquestador, debido al previsible aumento de complejidad de los procesos.

\begin{itemize}
    \item Despliegue de Apache Airflow como orquestador.
    \item Puesta al día de la documentación y memoria de los sprints 1, 2, 3 y 4
\end{itemize}

\subsection{Sprint 5 -  26/09/23 - 10/10/23}

\begin{itemize}
    \item .
    \item .
    \item .
    \item .
\end{itemize}


\section{Estudio de viabilidad}

\subsection{Viabilidad económica}

\subsection{Viabilidad legal}



\include{./tex/B_Requisitos}
\include{./tex/C_Diseno}
\include{./tex/D_Manual_programador}
\include{./tex/E_Manual_usuario}


\bibliographystyle{plain}
\bibliography{bibliografia}

\end{document}
