\capitulo{1}{Introducción}

En la era digital actual, el mercado de inversiones inmobiliarias en España está experimentando una transformación sin precedentes, principalmente por una combinación de factores económicos, sociales y tecnológicos. La búsqueda de inmuebles para comprar se ha vuelto cada vez más desafiante en un contexto donde los precios han visto un encarecimiento notable en los últimos años, lo que ha generado preocupaciones tanto para inversores como para aquellos que simplemente buscan un lugar para vivir. 

Según Idealista, el portal inmobiliario líder en España, el año 2023 terminará con un balance aparentemente contradictorio: mientras que las hipotecas disminuyen y el volumen de operaciones se resiente tras un año relevante como fue 2022, los precios de las viviendas continúan al alza, impulsados por una demanda que aún supera a una oferta cada vez más limitada~\cite{idealista_study}.

Este proyecto busca abordar precisamente estos desafíos, aplicando procesos de Ingeniería de Datos y técnicas de Aprendizaje Automático para optimizar la toma de decisiones en el ámbito de las inversiones inmobiliarias. Se pretende ofrecer una herramienta que no solo sea capaz de identificar oportunidades de inversión o búsqueda de nuevo hogar, sino también estudiar las fluctuaciones del mercado.

El proyecto se estructura en varias fases críticas, comenzando con la adquisición y preparación de datos a través de técnicas de \textit{scraping} y procesos ETL, seguido de la implementación de modelos de Aprendizaje Automático para evaluar el valor de las propiedades inmobiliarias en venta. La integración de estos datos enriquecidos en una aplicación web interactiva permite a los usuarios identificar inversiones inmobiliarias atractivas, además de estudiar la evolución del mercado inmobiliario.

Esta memoria también aborda los desafíos técnicos y metodológicos encontrados durante el desarrollo del proyecto, desde la selección de la fuente de datos hasta la optimización de los modelos de Aprendizaje Automático y la implementación de la solución en un entorno de producción. Además, se explora la relevancia de las tecnologías empleadas, remarcando por qué se han considerado para el proyecto.

Al finalizar, se presentan las conclusiones derivadas del trabajo, así como posibles líneas de investigación y desarrollo futuro para expandir y enriquecer aún más la solución propuesta. Por último, se incluyen Apéndices como el plan de proyecto, la guía de diseño de datos y arquitectura, el manual de programador y manual de usuario del sitio web resultante, disponible en \href{https://www.buscahogar.es}{buscahogar.es}.