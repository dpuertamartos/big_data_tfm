\capitulo{4}{Técnicas y herramientas}

Esta parte de la memoria tiene como objetivo presentar las técnicas metodológicas y las herramientas de desarrollo que se han utilizado para llevar a cabo el proyecto. Si se han estudiado diferentes alternativas de metodologías, herramientas, bibliotecas se puede hacer un resumen de los aspectos más destacados de cada alternativa, incluyendo comparativas entre las distintas opciones y una justificación de las elecciones realizadas. 
No se pretende que este apartado se convierta en un capítulo de un libro dedicado a cada una de las alternativas, sino comentar los aspectos más destacados de cada opción, con un repaso somero a los fundamentos esenciales y referencias bibliográficas para que el lector pueda ampliar su conocimiento sobre el tema.

\section{Scrappers}

Tabla comparativa de scrappers

\section{Base de datos primaria: mongodb}

Las secciones se incluyen con el comando section.

\section{Base de datos secundaria: SQLlite}

Explicar por qué el uso de una tan simple

\section{Herramientas de ETL}

Explicar por qué se decidió a usar python a secas en lugar de spark

\section{Orquestador: Apache airflow}

Explicar por qué se decidió el uso de airflow como orquestador

\section{Machine learning: scikit learn}

Las secciones se incluyen con el comando section.

\section{Aplicación web: Frontend - React}

Las secciones se incluyen con el comando section.

\section{Aplicación web: Backend - Node.js, express}

Las secciones se incluyen con el comando section.

\section{Despliegue - Oracle cloud}

Las secciones se incluyen con el comando section.
